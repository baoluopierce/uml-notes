%Contributor: Yixuan Li 
\section{Metric spaces, covers and simplicial complexes}
\subsection{Metric spaces.} A metric space $(M, \rho)$ is a set $M$ with a function $\rho: M \times M \rightarrow \mathbb{R}_+$, called a distance, such that for any $x,y,z \in M$,
\begin{enumerate}
    \item[i.] $\rho(x,y) \geq 0$ and $\rho(x,y)=0$ if and only if $x=y$.
    \item[ii.]$\rho(x,y) = \rho(y,x)$ and,
    \item[iii.]$\rho(x,z) \leq \rho(x,y)+\rho(y,z)$.
\end{enumerate}
Given a metric space $(M, \rho)$, the set $K(M)$ if its compact subsets can be endowed with the so-called Hausdorff distance: given two compact subsets, $A, B \subset M$, the Hausdorff distance $d_H(A,B) = $ the smallest non negative number $\delta$ such that for any $a\in A$, there exists $b\in B$ such that $\rho(a,b)\leq \delta$. And for any $b\in B$, there exists $a\in A$ such that $\rho(a,b)\leq \delta$. If any compact subset $C \subset M$, we denote by $d(\cdot, C): M \rightarrow \mathbb{R}_+$ the distance function to $C$ defined by $d(x,C):= \inf_{c\in C}\rho(x,c)$ for any $x\in M$, then we can prove that the Hausdorff distance between $A$ and $B$ is defined by any of the two following equalities.
\begin{align*}
    d_H(A,B) &= \max\{\sup_{b\in B}d(b,A), \sup_{a\in A}d(a,B)\}\\
    &= \sup_{x\in M}|d(x,A)-d(x,B)| = ||d(\cdot, A) - d(\cdot, B)||_\infty
\end{align*}
From a TDA perspective it provides a convenient way to quantify the proximity between different data sets issued from the same ambient metric space.However, it sometimes occurs in that one has to compare data set that are not sampled from the same ambient space. Fortunately, the notion of Hausdorff distance can be generalized to the comparison of any pair of compact metric spaces, giving rise to the notion of Gromov-Hausdorff distance. \\\\
\textbf{Definition. }The Gromov-Haudorff distance $d_{GH}(M_1, M_2)$ between two compact metric spaces is the infimum of the real numbers $r \geq 0$ such that there exists a metric space $(M, \rho)$ and two compacr subspaces $C_1, C_2 \subset M$ that are isometric to $M_1$ and $M_2$ and such that $d_H(C_1, C_2) \leq r$. \\\\
A central idea in TDA is to build higher dimensional equivalent of neighboring graphs by not only connecting pairs but also $(k + 1)$-uple of nearby data points. The resulting objects, called simplicial complexes, allow to identify new topological features such as cycles, voids and their higher dimensional counterpart.
\subsection{Geometric and abstract simplicial complexes.} Simplicial complexes can be seen as higher dimensional generalization of graphs. Given a set $\mathbb{X} = \{x_0,\cdots, x_k\} \subset \mathbb{R}^d$ of $k+1$ affinely independent points, the k-dimensional simplex $\sigma = [x_0,\cdots,x_k]$ spanned by $\mathbb{X}$ is the convex hull of $\mathbb{X}$. The points of $\mathbb{X}$ are called the vertices of $\sigma$ and the simplices spanned by the subsets of $\mathbb{X}$ are called the faces of $\sigma$. A geometric simplicial complex $K$ in $\mathbb{R}^d$ is a collection of simplices such that:
\begin{enumerate}
    \item[i.] face of a simplex of $K$ is a simplex of $K$
    \item[ii.] the intersection of any two simplices of $K$ is either empty or a common face of both.
\end{enumerate}
The union of the simplices of $K$ is a subset of $\mathbb{R}^d$ called the underlying space of K that inherits from the topology of $\mathbb{R}^d$. So, $K$ can also be seen as a topological space through its underlying
space. \\\\
Given a set $V$, an abstract simplicial complex with vertex set $V$ is a set $\Tilde{K}$ of finite subsets of $V$ such that the elements of $V$ belongs to $\Tilde{K}$ and for any $\sigma \in \Tilde{K}$ any subset of $\sigma$ belongs to $\Tilde{K}$. The elements of $\Tilde{K}$ are called the faces or the simplices of $\Tilde{K}$. The dimension of an abstract simplex is just its cardinality minus 1 and the dimension of $\Tilde{K}$is the largest dimension of its simplices. One can always associate to an abstract simplicial
complex $\Tilde{K}$, a topological space |$\Tilde{K}$| such that if $K$ is a geometric complex whose combinatorial description is the same as $\Tilde{K}$, then the underlying space of $K$ is homeomorphic to |$\Tilde{K}$|. Such a $K$ is called a geometric realization of $\Tilde{K}$. As a consequence, abstract simplicial complexes can be seen as topological spaces and geometric complexes can be seen as geometric realizations of their underlying combinatorial structure.
\subsection{Building simplicial complexes from data.} 
Assume that we are given a set of points $\mathbb{X}$ in a metric space $(M, \rho)$ and a real number $\alpha \geq $ 0. The Vietoris-Rips complex Rips$_\alpha(\mathbb{X})$ is the set of simplices $[x_0, \cdots, x_k]$ such that $d_{\mathbb{X}}(x_i, x_j) \leq \alpha$ for all $(i, j)$. Cech complex Cech$_\alpha(\mathbb{X})$ is defined as the set of simplices $[x_0,\cdots, x_k]$ such that $k+1$ closed balls $B(x_i,\alpha)$ have a non-empty intersection. The two complexes are related by \[\text{Rips}_\alpha(\mathbb{X}) \subset \text{Cech}_\alpha(\mathbb{X}) \subset \text{Rips}_{2\alpha}(\mathbb{X})\]
\subsection{The nerve theorem. }The Čech complex is a particular case of a family of complexes associated to covers. Given a cover $U = (U_i)_{i\in I}$ of $\mathbb{M}$, i.e. a family sets $U_i$ such that $\mathbb{M} = \cup_{i\in I}U_i$, the nerve of $U$ is the abstract simplicial complex $C(U)$ whose vertices are the $U_i$'s and such that \[\sigma = [U_{i_0},\cdots, U_{i_k}]\in C(U) \text{ if and only if } \cap_{j=0}^kU_{i_j}\neq \emptyset\]
Two topological spaces $X$ and $Y$ are usually considered as being the same from a topological point of view if they are homeomorphic, i.e. if there exist two continuous bijective maps $f : X \rightarrow Y$ and $g : Y \rightarrow X$ such that $f \circ g$ and $g \circ f$ are the identity map of $Y$ and $X$ respectively. Two continuous maps $f_0,f_1: X \rightarrow Y$ are said to be homotopic if there exists a continuous map $H: X\times [0,1] \rightarrow Y$ such that for any $x \in X, H(x,0) = f_0(x)$ and $H(x,1) = g(x)$. The spaces $X$ and $Y$ are then said to be homotopy equivalent if there exist two maps $f : X \rightarrow Y$ and $g : Y \rightarrow X$ such that $f \circ g$ and $g \circ f$ are homotopic to the identity map of $Y$ and $X$ respectively. $f$ and $g$ are then called homotopy equivalent. A space is said to be contractible if it is homotopy equivalent to a point.\\\\
\textbf{Theorem 1}(Nerve theorem). Let $U = (U_i)_{i\in I}$ be a cover of a topological space $X$ by open sets such that the intersection of any subcollection of the $U_i$'s is either empty or contractible. Then, $X$ and the nerve $C(U)$ are homotopy equivalent.\\\\
The Nerve Theorem plays a fundamental role in TDA: it provide a way to encode the topology
of continuous spaces into abstract combinatorial structures that are well-suited for the design of effective data structures and algorithms.